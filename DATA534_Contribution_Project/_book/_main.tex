% Options for packages loaded elsewhere
\PassOptionsToPackage{unicode}{hyperref}
\PassOptionsToPackage{hyphens}{url}
%
\documentclass[
]{book}
\usepackage{amsmath,amssymb}
\usepackage{lmodern}
\usepackage{iftex}
\ifPDFTeX
  \usepackage[T1]{fontenc}
  \usepackage[utf8]{inputenc}
  \usepackage{textcomp} % provide euro and other symbols
\else % if luatex or xetex
  \usepackage{unicode-math}
  \defaultfontfeatures{Scale=MatchLowercase}
  \defaultfontfeatures[\rmfamily]{Ligatures=TeX,Scale=1}
\fi
% Use upquote if available, for straight quotes in verbatim environments
\IfFileExists{upquote.sty}{\usepackage{upquote}}{}
\IfFileExists{microtype.sty}{% use microtype if available
  \usepackage[]{microtype}
  \UseMicrotypeSet[protrusion]{basicmath} % disable protrusion for tt fonts
}{}
\makeatletter
\@ifundefined{KOMAClassName}{% if non-KOMA class
  \IfFileExists{parskip.sty}{%
    \usepackage{parskip}
  }{% else
    \setlength{\parindent}{0pt}
    \setlength{\parskip}{6pt plus 2pt minus 1pt}}
}{% if KOMA class
  \KOMAoptions{parskip=half}}
\makeatother
\usepackage{xcolor}
\usepackage{color}
\usepackage{fancyvrb}
\newcommand{\VerbBar}{|}
\newcommand{\VERB}{\Verb[commandchars=\\\{\}]}
\DefineVerbatimEnvironment{Highlighting}{Verbatim}{commandchars=\\\{\}}
% Add ',fontsize=\small' for more characters per line
\usepackage{framed}
\definecolor{shadecolor}{RGB}{248,248,248}
\newenvironment{Shaded}{\begin{snugshade}}{\end{snugshade}}
\newcommand{\AlertTok}[1]{\textcolor[rgb]{0.94,0.16,0.16}{#1}}
\newcommand{\AnnotationTok}[1]{\textcolor[rgb]{0.56,0.35,0.01}{\textbf{\textit{#1}}}}
\newcommand{\AttributeTok}[1]{\textcolor[rgb]{0.77,0.63,0.00}{#1}}
\newcommand{\BaseNTok}[1]{\textcolor[rgb]{0.00,0.00,0.81}{#1}}
\newcommand{\BuiltInTok}[1]{#1}
\newcommand{\CharTok}[1]{\textcolor[rgb]{0.31,0.60,0.02}{#1}}
\newcommand{\CommentTok}[1]{\textcolor[rgb]{0.56,0.35,0.01}{\textit{#1}}}
\newcommand{\CommentVarTok}[1]{\textcolor[rgb]{0.56,0.35,0.01}{\textbf{\textit{#1}}}}
\newcommand{\ConstantTok}[1]{\textcolor[rgb]{0.00,0.00,0.00}{#1}}
\newcommand{\ControlFlowTok}[1]{\textcolor[rgb]{0.13,0.29,0.53}{\textbf{#1}}}
\newcommand{\DataTypeTok}[1]{\textcolor[rgb]{0.13,0.29,0.53}{#1}}
\newcommand{\DecValTok}[1]{\textcolor[rgb]{0.00,0.00,0.81}{#1}}
\newcommand{\DocumentationTok}[1]{\textcolor[rgb]{0.56,0.35,0.01}{\textbf{\textit{#1}}}}
\newcommand{\ErrorTok}[1]{\textcolor[rgb]{0.64,0.00,0.00}{\textbf{#1}}}
\newcommand{\ExtensionTok}[1]{#1}
\newcommand{\FloatTok}[1]{\textcolor[rgb]{0.00,0.00,0.81}{#1}}
\newcommand{\FunctionTok}[1]{\textcolor[rgb]{0.00,0.00,0.00}{#1}}
\newcommand{\ImportTok}[1]{#1}
\newcommand{\InformationTok}[1]{\textcolor[rgb]{0.56,0.35,0.01}{\textbf{\textit{#1}}}}
\newcommand{\KeywordTok}[1]{\textcolor[rgb]{0.13,0.29,0.53}{\textbf{#1}}}
\newcommand{\NormalTok}[1]{#1}
\newcommand{\OperatorTok}[1]{\textcolor[rgb]{0.81,0.36,0.00}{\textbf{#1}}}
\newcommand{\OtherTok}[1]{\textcolor[rgb]{0.56,0.35,0.01}{#1}}
\newcommand{\PreprocessorTok}[1]{\textcolor[rgb]{0.56,0.35,0.01}{\textit{#1}}}
\newcommand{\RegionMarkerTok}[1]{#1}
\newcommand{\SpecialCharTok}[1]{\textcolor[rgb]{0.00,0.00,0.00}{#1}}
\newcommand{\SpecialStringTok}[1]{\textcolor[rgb]{0.31,0.60,0.02}{#1}}
\newcommand{\StringTok}[1]{\textcolor[rgb]{0.31,0.60,0.02}{#1}}
\newcommand{\VariableTok}[1]{\textcolor[rgb]{0.00,0.00,0.00}{#1}}
\newcommand{\VerbatimStringTok}[1]{\textcolor[rgb]{0.31,0.60,0.02}{#1}}
\newcommand{\WarningTok}[1]{\textcolor[rgb]{0.56,0.35,0.01}{\textbf{\textit{#1}}}}
\usepackage{longtable,booktabs,array}
\usepackage{calc} % for calculating minipage widths
% Correct order of tables after \paragraph or \subparagraph
\usepackage{etoolbox}
\makeatletter
\patchcmd\longtable{\par}{\if@noskipsec\mbox{}\fi\par}{}{}
\makeatother
% Allow footnotes in longtable head/foot
\IfFileExists{footnotehyper.sty}{\usepackage{footnotehyper}}{\usepackage{footnote}}
\makesavenoteenv{longtable}
\usepackage{graphicx}
\makeatletter
\def\maxwidth{\ifdim\Gin@nat@width>\linewidth\linewidth\else\Gin@nat@width\fi}
\def\maxheight{\ifdim\Gin@nat@height>\textheight\textheight\else\Gin@nat@height\fi}
\makeatother
% Scale images if necessary, so that they will not overflow the page
% margins by default, and it is still possible to overwrite the defaults
% using explicit options in \includegraphics[width, height, ...]{}
\setkeys{Gin}{width=\maxwidth,height=\maxheight,keepaspectratio}
% Set default figure placement to htbp
\makeatletter
\def\fps@figure{htbp}
\makeatother
\setlength{\emergencystretch}{3em} % prevent overfull lines
\providecommand{\tightlist}{%
  \setlength{\itemsep}{0pt}\setlength{\parskip}{0pt}}
\setcounter{secnumdepth}{5}
\usepackage{booktabs}
\ifLuaTeX
  \usepackage{selnolig}  % disable illegal ligatures
\fi
\usepackage[]{natbib}
\bibliographystyle{plainnat}
\IfFileExists{bookmark.sty}{\usepackage{bookmark}}{\usepackage{hyperref}}
\IfFileExists{xurl.sty}{\usepackage{xurl}}{} % add URL line breaks if available
\urlstyle{same} % disable monospaced font for URLs
\hypersetup{
  pdftitle={A Minimal Book Example},
  pdfauthor={John Doe},
  hidelinks,
  pdfcreator={LaTeX via pandoc}}

\title{A Minimal Book Example}
\author{John Doe}
\date{2023-01-31}

\begin{document}
\maketitle

{
\setcounter{tocdepth}{1}
\tableofcontents
}
\hypertarget{about}{%
\chapter{About}\label{about}}

This is a \emph{sample} book written in \textbf{Markdown}. You can use anything that Pandoc's Markdown supports; for example, a math equation \(a^2 + b^2 = c^2\).

\hypertarget{usage}{%
\section{Usage}\label{usage}}

Each \textbf{bookdown} chapter is an .Rmd file, and each .Rmd file can contain one (and only one) chapter. A chapter \emph{must} start with a first-level heading: \texttt{\#\ A\ good\ chapter}, and can contain one (and only one) first-level heading.

Use second-level and higher headings within chapters like: \texttt{\#\#\ A\ short\ section} or \texttt{\#\#\#\ An\ even\ shorter\ section}.

The \texttt{index.Rmd} file is required, and is also your first book chapter. It will be the homepage when you render the book.

\hypertarget{render-book}{%
\section{Render book}\label{render-book}}

You can render the HTML version of this example book without changing anything:

\begin{enumerate}
\def\labelenumi{\arabic{enumi}.}
\item
  Find the \textbf{Build} pane in the RStudio IDE, and
\item
  Click on \textbf{Build Book}, then select your output format, or select ``All formats'' if you'd like to use multiple formats from the same book source files.
\end{enumerate}

Or build the book from the R console:

\begin{Shaded}
\begin{Highlighting}[]
\NormalTok{bookdown}\SpecialCharTok{::}\FunctionTok{render\_book}\NormalTok{()}
\end{Highlighting}
\end{Shaded}

To render this example to PDF as a \texttt{bookdown::pdf\_book}, you'll need to install XeLaTeX. You are recommended to install TinyTeX (which includes XeLaTeX): \url{https://yihui.org/tinytex/}.

\hypertarget{preview-book}{%
\section{Preview book}\label{preview-book}}

As you work, you may start a local server to live preview this HTML book. This preview will update as you edit the book when you save individual .Rmd files. You can start the server in a work session by using the RStudio add-in ``Preview book'', or from the R console:

\begin{Shaded}
\begin{Highlighting}[]
\NormalTok{bookdown}\SpecialCharTok{::}\FunctionTok{serve\_book}\NormalTok{()}
\end{Highlighting}
\end{Shaded}

\hypertarget{hello-bookdown}{%
\chapter{Hello bookdown}\label{hello-bookdown}}

All chapters start with a first-level heading followed by your chapter title, like the line above. There should be only one first-level heading (\texttt{\#}) per .Rmd file.

\hypertarget{a-section}{%
\section{A section}\label{a-section}}

All chapter sections start with a second-level (\texttt{\#\#}) or higher heading followed by your section title, like the sections above and below here. You can have as many as you want within a chapter.

\hypertarget{an-unnumbered-section}{%
\subsection*{An unnumbered section}\label{an-unnumbered-section}}
\addcontentsline{toc}{subsection}{An unnumbered section}

Chapters and sections are numbered by default. To un-number a heading, add a \texttt{\{.unnumbered\}} or the shorter \texttt{\{-\}} at the end of the heading, like in this section.

\hypertarget{parts}{%
\chapter{Parts}\label{parts}}

You can add parts to organize one or more book chapters together. Parts can be inserted at the top of an .Rmd file, before the first-level chapter heading in that same file.

Add a numbered part: \texttt{\#\ (PART)\ Act\ one\ \{-\}} (followed by \texttt{\#\ A\ chapter})

Add an unnumbered part: \texttt{\#\ (PART\textbackslash{}*)\ Act\ one\ \{-\}} (followed by \texttt{\#\ A\ chapter})

Add an appendix as a special kind of un-numbered part: \texttt{\#\ (APPENDIX)\ Other\ stuff\ \{-\}} (followed by \texttt{\#\ A\ chapter}). Chapters in an appendix are prepended with letters instead of numbers.

\hypertarget{hello-viji-here}{%
\chapter{Hello Viji here}\label{hello-viji-here}}

This has Viji's contributions in building API wrapper using openai and
indeed APIs.

\hypertarget{intro}{%
\section{01-Intro}\label{intro}}

All chapter sections start with a second-level (\texttt{\#\#}) or higher heading followed by your section title, like the sections above and below here. You can have as many as you want within a chapter.

\hypertarget{an-unnumbered-section-1}{%
\subsection*{An unnumbered section}\label{an-unnumbered-section-1}}
\addcontentsline{toc}{subsection}{An unnumbered section}

Chapters and sections are numbered by default. To un-number a heading, add a \texttt{\{.unnumbered\}} or the shorter \texttt{\{-\}} at the end of the heading, like in this section.

\hypertarget{parts-1}{%
\section{02 - Parts}\label{parts-1}}

\begin{longtable}[]{@{}
  >{\centering\arraybackslash}p{(\columnwidth - 8\tabcolsep) * \real{0.1429}}
  >{\raggedright\arraybackslash}p{(\columnwidth - 8\tabcolsep) * \real{0.2000}}
  >{\raggedright\arraybackslash}p{(\columnwidth - 8\tabcolsep) * \real{0.2571}}
  >{\raggedright\arraybackslash}p{(\columnwidth - 8\tabcolsep) * \real{0.2000}}
  >{\raggedright\arraybackslash}p{(\columnwidth - 8\tabcolsep) * \real{0.2000}}@{}}
\toprule()
\begin{minipage}[b]{\linewidth}\centering
Date
\end{minipage} & \begin{minipage}[b]{\linewidth}\raggedright
Module
\end{minipage} & \begin{minipage}[b]{\linewidth}\raggedright
Activity
\end{minipage} & \begin{minipage}[b]{\linewidth}\raggedright
Outcome/Achievements
\end{minipage} & \begin{minipage}[b]{\linewidth}\raggedright
Additional info
\end{minipage} \\
\midrule()
\endhead
13-Jan-2023 & NA & Initiation & Research on API and brainstorm with team & \\
14,15-Jan-2023 & NA & Initiation & Research on API and brainstorm with team & \\
16-Jan-2023 & NA & Initiation & Group discussion on next steps & \\
16-Jan-2023 & NA & Initiation & Contribution log creation and logging & \\
17-Jan-2023 & NA & Initiation & Researched different APIs available & \\
17-Jan-2023 & NA & Initiation & Confirmation on using openAI & \\
18-Jan-2023 & NA & Proposal & Created draft proposal & \\
20-Jan-2023 & NA & Proposal & Finished discussions and review on proposal. We considered using RapidAPI & \\
& & & but dropped as we are not sure of the permissions and terms that come with it & \\
21-Jan-2023 & & Design & Looking at next steps for designing the wrapper & \\
\bottomrule()
\end{longtable}

\hypertarget{references}{%
\section{03-references}\label{references}}

Add all outside references and biliography links

\begin{longtable}[]{@{}cll@{}}
\toprule()
Date & Link & Comments \\
\midrule()
\endhead
abc & & \\
\bottomrule()
\end{longtable}

\hypertarget{dependencies-open-questions-and-comments}{%
\section{04-dependencies, open questions and comments}\label{dependencies-open-questions-and-comments}}

Anything to communicate with other project members or consider for future.

\begin{longtable}[]{@{}clll@{}}
\toprule()
Date & Module & Description & Comments \\
\midrule()
\endhead
15-Jan-2023 & & & \\
\bottomrule()
\end{longtable}

\hypertarget{hi-avishek-here}{%
\chapter{Hi Avishek here}\label{hi-avishek-here}}

This includes Avishek's contribution into developing an API wrapper utilising OpenAI and Indeed APIs.

\hypertarget{intro-1}{%
\section{01-Intro}\label{intro-1}}

All chapter sections start with a second-level (\texttt{\#\#}) or higher heading followed by your section title, like the sections above and below here. You can have as many as you want within a chapter.

\hypertarget{an-unnumbered-section-2}{%
\subsection*{An unnumbered section}\label{an-unnumbered-section-2}}
\addcontentsline{toc}{subsection}{An unnumbered section}

Chapters and sections are numbered by default. To un-number a heading, add a \texttt{\{.unnumbered\}} or the shorter \texttt{\{-\}} at the end of the heading, like in this section.

\hypertarget{parts-2}{%
\section{02 - Parts}\label{parts-2}}

\begin{longtable}[]{@{}
  >{\centering\arraybackslash}p{(\columnwidth - 8\tabcolsep) * \real{0.1429}}
  >{\raggedright\arraybackslash}p{(\columnwidth - 8\tabcolsep) * \real{0.2000}}
  >{\raggedright\arraybackslash}p{(\columnwidth - 8\tabcolsep) * \real{0.2571}}
  >{\raggedright\arraybackslash}p{(\columnwidth - 8\tabcolsep) * \real{0.2000}}
  >{\raggedright\arraybackslash}p{(\columnwidth - 8\tabcolsep) * \real{0.2000}}@{}}
\toprule()
\begin{minipage}[b]{\linewidth}\centering
Date
\end{minipage} & \begin{minipage}[b]{\linewidth}\raggedright
Module
\end{minipage} & \begin{minipage}[b]{\linewidth}\raggedright
Activity
\end{minipage} & \begin{minipage}[b]{\linewidth}\raggedright
Outcome/Achievements
\end{minipage} & \begin{minipage}[b]{\linewidth}\raggedright
Additional info
\end{minipage} \\
\midrule()
\endhead
13-Jan-2023 & NA & Initiation & Researched on API and brainstorm with team & \\
14-Jan-2023 & NA & Initiation & Read Documentation of OpenAi API \& test the api & \\
16-Jan-2023 & NA & Initiation & Group discussion on next steps & \\
16-Jan-2023 & NA & Initiation & Created Github Repo & \\
17-Jan-2023 & NA & Initiation & Applied for Indeed API publisher number & \\
17-Jan-2023 & NA & Initiation & Confirmation on using openAI & \\
18-Jan-2023 & NA & Initiation & Tested LinkedIn RapidAPI & \\
18-Jan-2023 & NA & Initiation & Checked availability of glassdoor API & \\
19-Jan-2023 & NA & Initiation & Tried to scrape LinkedIn & \\
20-Jan-2023 & NA & Proposal & Review Proposal & \\
\bottomrule()
\end{longtable}

\hypertarget{references-1}{%
\section{03-references}\label{references-1}}

Add all outside references and biliography links

\begin{longtable}[]{@{}cll@{}}
\toprule()
Date & Link & Comments \\
\midrule()
\endhead
abc & & \\
\bottomrule()
\end{longtable}

\hypertarget{dependencies-open-questions-and-comments-1}{%
\section{04-dependencies, open questions and comments}\label{dependencies-open-questions-and-comments-1}}

Anything to communicate with other project members or consider for future.

\begin{longtable}[]{@{}clll@{}}
\toprule()
Date & Module & Description & Comments \\
\midrule()
\endhead
15-Jan-2023 & & & \\
\bottomrule()
\end{longtable}

\hypertarget{hi-noman-here}{%
\chapter{Hi Noman here}\label{hi-noman-here}}

This includes Noman's contribution into developing an API wrapper utilising OpenAI and Indeed APIs.

\hypertarget{intro-2}{%
\section{01-Intro}\label{intro-2}}

All chapter sections start with a second-level (\texttt{\#\#}) or higher heading followed by your section title, like the sections above and below here. You can have as many as you want within a chapter.

\hypertarget{an-unnumbered-section-3}{%
\subsection*{An unnumbered section}\label{an-unnumbered-section-3}}
\addcontentsline{toc}{subsection}{An unnumbered section}

Chapters and sections are numbered by default. To un-number a heading, add a \texttt{\{.unnumbered\}} or the shorter \texttt{\{-\}} at the end of the heading, like in this section.

\hypertarget{parts-3}{%
\section{02 - Parts}\label{parts-3}}

\begin{longtable}[]{@{}
  >{\centering\arraybackslash}p{(\columnwidth - 8\tabcolsep) * \real{0.1429}}
  >{\raggedright\arraybackslash}p{(\columnwidth - 8\tabcolsep) * \real{0.2000}}
  >{\raggedright\arraybackslash}p{(\columnwidth - 8\tabcolsep) * \real{0.2571}}
  >{\raggedright\arraybackslash}p{(\columnwidth - 8\tabcolsep) * \real{0.2000}}
  >{\raggedright\arraybackslash}p{(\columnwidth - 8\tabcolsep) * \real{0.2000}}@{}}
\toprule()
\begin{minipage}[b]{\linewidth}\centering
Date
\end{minipage} & \begin{minipage}[b]{\linewidth}\raggedright
Module
\end{minipage} & \begin{minipage}[b]{\linewidth}\raggedright
Activity
\end{minipage} & \begin{minipage}[b]{\linewidth}\raggedright
Outcome/Achievements
\end{minipage} & \begin{minipage}[b]{\linewidth}\raggedright
Additional info
\end{minipage} \\
\midrule()
\endhead
13-Jan-2023 & NA & Initiation & Explored project suggestions and use cases for OpenAI & \\
14-Jan-2023 & NA & Initiation & Further exploring openAI and brainstorming with team & \\
16-Jan-2023 & NA & Initiation & Group discussion on next steps & \\
17-Jan-2023 & NA & Initiation & Confirmation on using openAI with team & \\
17-Jan-2023 & NA & Initiation & Explored Indeed and its scraping potential & \\
18-Jan-2023 & NA & Proposal & Explored Indeed and its scraping potential for interview questions & \\
18-Jan-2023 & NA & Proposal & Researched OpenAI's capabilities to curate interview questions & \\
19-Jan-2023 & NA & Proposal & Met with team to further discuss scope and functionality & \\
20-Jan-2023 & NA & & Proposal & Read project proposal \\
27-Jan-2023 & NA & & Project & played around with openAI api and the temperature parameter \\
28-Jan-2023 & NA & & Project & Discussed progress and any roadblocks with group \\
29-Jan-2023 & NA & & Project & Worked on interview questions wrapper function \\
\bottomrule()
\end{longtable}

\hypertarget{references-2}{%
\section{03-references}\label{references-2}}

Add all outside references and biliography links

\begin{longtable}[]{@{}cll@{}}
\toprule()
Date & Link & Comments \\
\midrule()
\endhead
abc & & \\
\bottomrule()
\end{longtable}

\hypertarget{dependencies-open-questions-and-comments-2}{%
\section{04-dependencies, open questions and comments}\label{dependencies-open-questions-and-comments-2}}

Anything to communicate with other project members or consider for future.

\begin{longtable}[]{@{}clll@{}}
\toprule()
Date & Module & Description & Comments \\
\midrule()
\endhead
15-Jan-2023 & & & \\
\bottomrule()
\end{longtable}

  \bibliography{book.bib,packages.bib}

\end{document}
